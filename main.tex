\documentclass[output=inprep,
  nonflat,
  modfonts,
%  colorlinks,
  showindex,
  draftmode
]{langsci/langscibook}\usepackage[]{graphicx}\usepackage[]{color}
%% maxwidth is the original width if it is less than linewidth
%% otherwise use linewidth (to make sure the graphics do not exceed the margin)
\makeatletter
\def\maxwidth{ %
  \ifdim\Gin@nat@width>\linewidth
    \linewidth
  \else
    \Gin@nat@width
  \fi
}
\makeatother

\usepackage{Sweavel}


  
\title{Probabilistic German Morphosyntax}
\subtitle{}
\BackTitle{Probabilistic German Morphosyntax}
\BackBody{\textit{Probabilistic German Morphosyntax} is the first outline of a systematic account of German morphosyntax and graphemics in the framework of probabilistic grammar.}
\dedication{}
\typesetter{Roland Schäfer}
\proofreader{Elizabeth Pankratz}
\author{Roland Schäfer (with contributions by Ulrike Sayatz and Elizabeth Pankratz)}
% \BookDOI{}
\renewcommand{\lsISBNdigital}{000-0-000000-00-0}
\renewcommand{\lsISBNhardcover}{000-0-000000-00-0}
\renewcommand{\lsISBNsoftcover}{000-0-000000-00-0}
\renewcommand{\lsISBNsoftcoverus}{000-0-000000-00-0}
\renewcommand{\lsSeries}{tbls}
\renewcommand{\lsSeriesNumber}{99}
% \renewcommand{\lsURL}{http://langsci-press.org/catalog/book/000}
  

\usepackage{knitr}

\usepackage{tabularx} 
\usepackage{longtable}

\usepackage{./langsci/styles/langsci-optional}
\usepackage{./langsci/styles/langsci-gb4e}
\usepackage{./langsci/styles/langsci-lgr}
\usepackage{./langsci/styles/langsci-glyphs}
\usepackage{./langsci/styles/langsci-tbls}

\usepackage[english]{babel}

% \usepackage[hang,flushmargin]{footmisc}
% \setlength\footnotemargin{10pt}

\usepackage{listings}

\usepackage{unicode-math}

\usepackage{csquotes}
\usepackage{fontspec}

\usepackage{enumitem}


%% hyphenation points for line breaks
%% Normally, automatic hyphenation in LaTeX is very good
%% If a word is mis-hyphenated, add it to this file
%%
%% add information to TeX file before \begin{document} with:
%% %% hyphenation points for line breaks
%% Normally, automatic hyphenation in LaTeX is very good
%% If a word is mis-hyphenated, add it to this file
%%
%% add information to TeX file before \begin{document} with:
%% %% hyphenation points for line breaks
%% Normally, automatic hyphenation in LaTeX is very good
%% If a word is mis-hyphenated, add it to this file
%%
%% add information to TeX file before \begin{document} with:
%% \include{localhyphenation}
\hyphenation{
affri-ca-te
affri-ca-tes
com-ple-ments
}
\hyphenation{
affri-ca-te
affri-ca-tes
com-ple-ments
}
\hyphenation{
affri-ca-te
affri-ca-tes
com-ple-ments
}
\bibliography{localbibliography} 

\usepackage{lipsum}

\setmonofont{Inconsolatar.ttf}

\definecolor{listingbackground}{gray}{0.95}
\lstdefinestyle{RStyle}{
  language=R,
  basicstyle=\ttfamily\footnotesize,
  keywordstyle=\ttfamily\bfseries\color{lsDarkOrange},
  stringstyle=\ttfamily\color{lsDarkBlue},
  identifierstyle=\ttfamily\color{lsDarkGreenOne},
  commentstyle=\ttfamily\color{lsLightBlue},
  upquote=true,
  breaklines=true,
  backgroundcolor=\color{listingbackground},
  framesep=5mm,
  frame=trlb,
  framerule=0pt,
  linewidth=\dimexpr\textwidth-5mm,
  xleftmargin=5mm
  %numbers=left, numberstyle=\color{lsLightGray}, stepnumber=1, numbersep=10pt
  }
  
\lstset{style=Rstyle}


\begin{document}     

% By LSP.
\renewbibmacro*{index:name}[5]{%
  \usebibmacro{index:entry}{#1}
    {\iffieldundef{usera}{}{\thefield{usera}\actualoperator}\mkbibindexname{#2}{#3}{#4}{#5}}}


% By LSP.
\makeatletter
\def\blx@maxline{77}
\makeatother


% Fix line spacing in list environmens.
\setlist{noitemsep}


% Correct hyperref colors which otherwise give you eye cancer.
\hypersetup{
  linkbordercolor  = {white}
  , linkcolor        = {lsMidDarkBlue}
  , anchorcolor      = {lsMidWine}
  , citecolor        = {lsDarkGreenOne}
  , menucolor        = {lsMidDarkBlue}
  , urlcolor         = {lsDarkOrange}
%    , filecolor       = {}
%    , runcolor        = {}
}


% Use a better mono font, ideal for code.
% https://github.com/chrissimpkins/codeface/tree/master/fonts/inconsolata-g
\setmonofont{Inconsolata-g}


% Use a math font that actually works! Requires unicode-math paackage.
% https://github.com/khaledhosny/libertinus
\setmathfont[Scale=MatchUppercase]{libertinusmath-regular.otf}


% Set listing style. knitr uses RStyle style. Which you have to know...
\definecolor{listingbackground}{gray}{0.95}
\lstdefinestyle{RStyle}{
  language=R,
  basicstyle=\ttfamily\footnotesize,
  keywordstyle=\ttfamily\color{lsDarkOrange},
  stringstyle=\ttfamily\color{lsDarkBlue},
  identifierstyle=\ttfamily\color{lsDarkGreenOne},
  commentstyle=\ttfamily\color{lsLightBlue},
  upquote=true,
  breaklines=true,
  backgroundcolor=\color{listingbackground},
  framesep=5mm,
  frame=trlb,
  framerule=0pt,
  linewidth=\dimexpr\textwidth-5mm,
  xleftmargin=5mm
  }
\lstset{style=Rstyle}

\newcommand{\eg}{e.\,g.,\ }
\newcommand{\Eg}{E.\,g.,\ }
\newcommand{\ie}{i.\,e.,\ }
\newcommand{\Ie}{I.\,e.,\ }
\newcommand{\wrt}{w.\,r.\,t.\ }
\newcommand{\Wrt}{W.\,r.\,t.\ }

\newcommand{\Sub}[1]{\ensuremath{_\textrm{#1}}}
\newcommand{\Sup}[1]{\ensuremath{^\textrm{#1}}}
\newcommand{\Subsf}[1]{\ensuremath{_\textsf{#1}}}
\newcommand{\Supsf}[1]{\ensuremath{^\textsf{#1}}}
\newcommand{\pPB}{p\Sub{PB}}
\newcommand{\mpPB}{\ensuremath{p_{\textrm{PB}}}}

\newcommand{\NACb}{NAC\Sub{bare}}
\newcommand{\NACa}{NAC\Sub{adj}}
\newcommand{\PGCd}{PGC\Sub{det}}
\newcommand{\PGCa}{PGC\Sub{adj}}

\newcommand{\RDefArt}{(Chapter~\ref{sec:cliticisationandparadigms})}
\newcommand{\RWeil}{(Chapter~\ref{sec:clausesandphrases})}
\newcommand{\RWeakN}{(Chapter~\ref{sec:paradigmsandprototypes})}
\newcommand{\RMeasure}{(Chapter~\ref{sec:constructionsandprototypes})}

\newcommand{\RADefArt}{Chapter~\ref{sec:cliticisationandparadigms}}
\newcommand{\RAWeil}{Chapter~\ref{sec:clausesandphrases}}
\newcommand{\RAWeakN}{Chapter~\ref{sec:paradigmsandprototypes}}
\newcommand{\RAMeasure}{Chapter~\ref{sec:constructionsandprototypes}}

\newcommand{\RODefArt}{Chapter~\ref{sec:cliticisationandparadigms}}
\newcommand{\ROWeil}{Chapter~\ref{sec:clausesandphrases}}
\newcommand{\ROWeakN}{Chapter~\ref{sec:paradigmsandprototypes}}
\newcommand{\ROMeasure}{Chapter~\ref{sec:constructionsandprototypes}}
 
 
\maketitle                
\frontmatter

\currentpdfbookmark{Contents}{name} 
\tableofcontents
\addchap{Preface}

\begin{quote}
  Thus, [what we do] may, because of the neglect of other important structural properties, be to classify natural language along an ultimately irrelevant dimension. \parencite[436--437]{ParteeEa1990}
\end{quote}


\addchap{Acknowledgments}

First and foremost, I want to thank Ulrike Sayatz.
Ulrike convinced me that graphemics is a highly relevant discipline and an integral and indispensible part of linguistics and grammar.
Together, we developed the ideas of \textit{usage-based graphemics}, which underlie some of the work presented here.
Ulrike also helped me to understand German morphosyntax much better through many discussions.
Most importantly, she has been a productive and inspiring co-author for many years now.

I thank Elizabeth Pankratz for being an inspiring collaborator, co-author and strong moral supporter.
Among other things, Elizabeth changed the way I look at compounds, leading to significant improvements in the DECOW corpus and fascinating joint research on compounds.

Furthermore, I am grateful to Felix Bildhauer for ongoing discussions about corpus construction, corpus analysis, and statistics (worth at least a gazillion Złoty).
Without our collaboration, the COW corpora would not exist, and I could not have conducted the research presented in this book.

For their substantial support throughout my career as teachers, colleagues, employers, thesis advisors, and moral supporters (or any combination thereof), I am indebted to (in alphabetical order) Götz Keydana, Michael Job, Stefan Müller, Manfred Sailer, Gert Webelhuth.
Without any single one of them, I would not have made it this far.

Student assistants who contributed significantly to the success of my research by doing a lot of painstaking work on corpus data and supervising experiments are (in alphabetical order) Sarah Dietzfelbinger, Lea Helmers, Kim Maser, and Luise Rißmann.
I am truly grateful to them.

Finally and most importantly, I want to thank everyone who supported me on a personal level over the years, including but surely not restricted to (in alphabetical order) Matthias B.\ Döring, Tanja Hagedorn, Julia Schmidt, and my parents.

My work on this project was funded in part by the \textit{Deutsche Forschungsgemeinschaft} (DFG, personal grant SCHA1916/1-1) through the project \textit{Linguistic Web Characterisation}.

\addchap{Abbreviations and symbols}

\section*{Abbreviations}

\begin{longtable}{p{0.1\textwidth}p{0.9\textwidth}}
  ANOVA & analysis of variance \\
  CDF   & cumulative distribution function \\
  CLT   & central limit theorem \\
  cp.   & ceteris paribus (all other things being equal) \\
  iid.  & independent and identically distributed \\
  LM    & linear model \\
  LMM   & linear mixed model \\
  GLM   & linear mixed model \\
  GLMM  & generalised linear mixed model \\
  PDF   & probability density function \\
  VCOV  & variance-covariance matrix \\
\end{longtable}


\section*{Symbols}

Symbols are overloaded ad-hoc to denote either a (possibly indexed) value such as $s_x=1$ (for ``the population mean of variable $x$ is $1$'') or a function such as $s(x)=1$ where applicable.

\begin{longtable}{p{0.1\textwidth}p{0.9\textwidth}} 

		    &  \textbf{Mathematical symbols} \\

  % Pure symbols
  $x\thicksim D$    & \textit{x follows D} ($x$ a variable, $D$ a distribution) \\
  $\bar{x}$         & sample arithmetic mean of $x$\\
  $\tilde{x}$       & sample median of $x$\\
  $\hat{x}$         & predicted value of $x$\\
 
                    & \\
		    &  \textbf{Letter-like symbols} \\

  % Letter-like.
  $\alpha$          & alpha level \\
  $\alpha_i$        & intercept $i$ \\
  $\beta$           & beta level \\
  $\beta_i$         & first-level coefficient $i$ \\
  $df$              & degrees of freedom \\
  $e$               & Euler constant \\
  $\epsilon$        & observation-level error \\
  $f$               & frequency \\
  $F$               & F statistic (see ANOVA)\\
  $\gamma_i$        & second-level coefficient $i$ \\
  $H$               & Kruskal-Wallis statistic \\
  $H_0$             & null hypothesis \\
  $H_A$             & alternative hypothesis \\
  $M_M$             & main hypothesis \\
  $IQR$             & inter-quartile range \\
  $\mathcal{L}$     & Likelihood \\
  $\mu$             & population mean \\
  $\mu_i$           & mean of modeled effect $i$ \\
  $n$               & sample size \\
  $N$               & population size \\
  $O$               & Odds \\
  $p$               & proportion \\
  $P_i$             & the $i$-th percentile \\
  $Pr$              & probability \\
  $\varphi$         & dispersion parameter \\
  $Q_i$             & $i$-th quartile \\
  $r$               & sample covariance coefficient \\
  $r^2$             & coefficient of determination \\
  $R^2$             & multifactorial coefficient of determination \\
  $\rho$            & population covariance coefficient \\
  $s$               & sample standard deviation of $x$ \\
  $s^2$             & sample variance of $x$ \\ 
  $SE$              & standard error \\
  $SS$              & sum of squares \\
  $\sigma$          & population standard deviation \\
  $\sigma^2$        & population variance \\ 
  $U$               & Mann-Whitney statistic \\
  $\chi^2$          & chi square statistic \\
\end{longtable} 

\vspace{\baselineskip}
\noindent Random distributions are denoted by bold-printed abbreviated names instead of the incoherent symbols sometimes used.
\vspace{\baselineskip}

\begin{longtable}{p{0.1\textwidth}p{0.9\textwidth}}
  $\mathbf{Bern}$   & Bernoulli distribution \\
  $\mathbf{Exp}$    & exponential distribution \\
  $\mathbf{F}$      & $F$ distribution \\
  $\mathbf{Norm}$   & normal (Gaussian) distribution \\
  $\mathbf{t}$      & t distribution \\
  $\mathbf{Unif}$   & uniform distribution \\
  $\mathbf{Chisq}$  & $\chi^2$ distribution \\
\end{longtable} 
 
\mainmatter         




\chapter{Science, data, and statistics}
\label{sec:sciencedataandstatistics}

\cite{MacDonaldGardner2000}

\tblssy[lsLightGray]{glass}{Test SY}{\lipsum[66]}

\tblsli[lsLightGray]{1}{Test LI}{\lipsum[66]}

\tblsfi[lsYellow]{Test FI}{\lipsum[66]}

\tblsfr[lsYellow]{glass}{Test FR}{\lipsum[66]}

\tblsfd{lsLightGray}{1}{Test FD}{\lipsum[66]}

\pagebreak

In this book, code listing are displayed as inline blocks such as the following simple code which simulates t-tests under the null hypothesis in order to demonstrate that all p-values have equal probability under the null.

\begin{Schunk}
\begin{Sinput}
# Set simulation parameters.
nsim  <- 1000
n     <- 100
meen  <- 0
stdev <- 1

# Data structure for results.
sims <- rep(NA, nsim)

# Simulations.
for (i in 1:nsim) {
  a <- rnorm(n, mean = meen, sd = stdev)
  b <- rnorm(n, mean = meen, sd = stdev)
  p <- t.test(a,b)$p.value
  sims[i] <- p
}
\end{Sinput}
\end{Schunk}


\begin{Schunk}
\begin{figure}[H]

{\centering \includegraphics[width=\maxwidth]{/Users/user/Workingcopies/SMIL/figuressimttestscatter-1} 

}

\caption[Scatterplot of p-values]{Scatterplot of p-values.}\label{fig:simttestscatter}
\end{figure}
\end{Schunk}

\begin{Schunk}
\begin{figure}[H]

{\centering \includegraphics[width=\maxwidth]{/Users/user/Workingcopies/SMIL/figuressimttestecdf-1} 

}

\caption[Empirical cumulative density distribution of p-values]{Empirical cumulative density distribution of p-values.}\label{fig:simttestecdf}
\end{figure}
\end{Schunk}

\begin{Schunk}
\begin{figure}[H]

{\centering \includegraphics[width=\maxwidth]{/Users/user/Workingcopies/SMIL/figuressimttesthist-1} 

}

\caption[Histogram of p-values]{Histogram of p-values.}\label{fig:simttesthist}
\end{figure}
\end{Schunk}

See Figure~\ref{fig:simttestecdf} for the cumulative density of p-values under the null in a series of 1000 t-tests.
This was plotted using the following command.

\begin{Schunk}
\begin{Sinput}
plot(ecdf(sims))
\end{Sinput}
\end{Schunk}


\chapter{Describing data}
\label{sec:describingdata}


\chapter{Visualising data}
\label{sec:visualisingdata}


\chapter{Tests}
\label{sec:tests}


\chapter{Models}
\label{sec:models}


\chapter{Generalised models}
\label{sec:generalisedmodels}


\chapter{Mixed models}
\label{sec:mixedmodels}




\chapter{Where to go from here?}
\label{sec:wheretogofromhere}

% \is{some term| see {some other term}}
\il{some language| see {some other language}}
\issa{some term with pages}{some other term also of interest}
\ilsa{some language with pages}{some other lect also of interest} 
\backmatter
\phantomsection%this allows hyperlink in ToC to work
\printbibliography[heading=references] 
\cleardoublepage

\phantomsection 
\addcontentsline{toc}{chapter}{Index} 
\addcontentsline{toc}{section}{Name index}
\ohead{Name index} 
\printindex 
\cleardoublepage
  
\phantomsection 
\addcontentsline{toc}{section}{Language index}
\ohead{Language index} 
\printindex[lan] 
\cleardoublepage
  
\phantomsection 
\addcontentsline{toc}{section}{Subject index}
\ohead{Subject index} 
\printindex[sbj]
\ohead{} 
 
\end{document} 

