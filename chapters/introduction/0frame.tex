\documentclass[a4paper, biblatex, charis, linguex]{glossa}

\bibliography{rs} 

\usepackage{sectsty} 
\allsectionsfont{\normalfont\sffamily\bfseries}
\subsectionfont{\normalfont\sffamily\bfseries\itshape}
\let\B\relax 
\let\T\relax
\usepackage[linguistics]{forest}
\usepackage{ragged2e}

% \usepackage[firstpage]{draftwatermark}
% \SetWatermarkFontSize{24pt}
% \SetWatermarkHorCenter{300pt}
% \SetWatermarkColor[rgb]{1,0,0}
% \SetWatermarkText{\textsf{\parbox{18cm}{\centering%
%                 This is not necessarily the version submitted to\\
% 		the Faculty of Languages and Letters\\
% 	        of Humboldt University Berlin.\\[\baselineskip]
% 	        This version was compiled on \today.}}}


\pdfauthor{Roland Schäfer}
\pdftitle{Probabilistic German Morphosyntax}
\pdfkeywords{probabilistic grammar, morpho-syntax, alternations, corpus and experimental data, German}

\title{Probabilistic German Morphosyntax}

\author{
  \spauthor{Roland Schäfer\\ 
  \small{Nürnberger Straße 45\\
  10789 Berlin\\
  mail@rolandschaefer.net}
  }
}

\usepackage{booktabs}
\usepackage{rotating}

% By LSP.
\renewbibmacro*{index:name}[5]{%
  \usebibmacro{index:entry}{#1}
    {\iffieldundef{usera}{}{\thefield{usera}\actualoperator}\mkbibindexname{#2}{#3}{#4}{#5}}}


% By LSP.
\makeatletter
\def\blx@maxline{77}
\makeatother


% Fix line spacing in list environmens.
\setlist{noitemsep}


% Correct hyperref colors which otherwise give you eye cancer.
\hypersetup{
  linkbordercolor  = {white}
  , linkcolor        = {lsMidDarkBlue}
  , anchorcolor      = {lsMidWine}
  , citecolor        = {lsDarkGreenOne}
  , menucolor        = {lsMidDarkBlue}
  , urlcolor         = {lsDarkOrange}
%    , filecolor       = {}
%    , runcolor        = {}
}


% Use a better mono font, ideal for code.
% https://github.com/chrissimpkins/codeface/tree/master/fonts/inconsolata-g
\setmonofont{Inconsolata-g}


% Use a math font that actually works! Requires unicode-math paackage.
% https://github.com/khaledhosny/libertinus
\setmathfont[Scale=MatchUppercase]{libertinusmath-regular.otf}


% Set listing style. knitr uses RStyle style. Which you have to know...
\definecolor{listingbackground}{gray}{0.95}
\lstdefinestyle{RStyle}{
  language=R,
  basicstyle=\ttfamily\footnotesize,
  keywordstyle=\ttfamily\color{lsDarkOrange},
  stringstyle=\ttfamily\color{lsDarkBlue},
  identifierstyle=\ttfamily\color{lsDarkGreenOne},
  commentstyle=\ttfamily\color{lsLightBlue},
  upquote=true,
  breaklines=true,
  backgroundcolor=\color{listingbackground},
  framesep=5mm,
  frame=trlb,
  framerule=0pt,
  linewidth=\dimexpr\textwidth-5mm,
  xleftmargin=5mm
  }
\lstset{style=Rstyle}

\newcommand{\eg}{e.\,g.,\ }
\newcommand{\Eg}{E.\,g.,\ }
\newcommand{\ie}{i.\,e.,\ }
\newcommand{\Ie}{I.\,e.,\ }
\newcommand{\wrt}{w.\,r.\,t.\ }
\newcommand{\Wrt}{W.\,r.\,t.\ }

\newcommand{\Sub}[1]{\ensuremath{_\textrm{#1}}}
\newcommand{\Sup}[1]{\ensuremath{^\textrm{#1}}}
\newcommand{\pPB}{p\Sub{PB}}
\newcommand{\mpPB}{\ensuremath{p_{\textrm{PB}}}}

\newcommand{\RDefArt}{(Chapter~\ref{sec:cliticisationandparadigms})}
\newcommand{\RWeil}{(Chapter~\ref{sec:clausesandphrases})}
\newcommand{\RWeakN}{(Chapter~\ref{sec:paradigmsandprototypes})}
\newcommand{\RMeasure}{(Chapter~\ref{sec:constructionsandprototypes})}

\newcommand{\RADefArt}{Chapter~\ref{sec:cliticisationandparadigms}}
\newcommand{\RAWeil}{Chapter~\ref{sec:clausesandphrases}}
\newcommand{\RAWeakN}{Chapter~\ref{sec:paradigmsandprototypes}}
\newcommand{\RAMeasure}{Chapter~\ref{sec:constructionsandprototypes}}

\newcommand{\RODefArt}{Chapter~\ref{sec:cliticisationandparadigms}}
\newcommand{\ROWeil}{Chapter~\ref{sec:clausesandphrases}}
\newcommand{\ROWeakN}{Chapter~\ref{sec:paradigmsandprototypes}}
\newcommand{\ROMeasure}{Chapter~\ref{sec:constructionsandprototypes}}


\begin{document}

\thispagestyle{empty}

\centering
\sffamily
{\Large\textbf{Probabilistic German Morphosyntax}}\\[\baselineskip]

{\large HABILITATIONSSCHRIFT\\
  zur Erlangung der Lehrbefähigung für das Fach\\
  Germanistische und Allgemeine Sprachwissenschaft\\[0.5\baselineskip]

  vorgelegt 
  der Philosophischen Fakultät II\\
  der Humboldt-Universität zu Berlin\\[0.5\baselineskip]
  
  von\\
  Dr.\ Roland Schäfer\\
  geb.\ am 06. Janur 1974 in Düsseldorf\\[3\baselineskip]

  \raggedright
  \large
  Präsidentin der Humboldt-Universität zu Berlin:\\[0.25\baselineskip]
  Prof.\ Dr.-Ing.\ Dr.\ Sabine Kunst
  \vspace{1.5\baselineskip}

  Dekanin:\\[0.25\baselineskip]
  Prof.\ Dr.\ Ulrike Vedder
  \vspace{1.5\baselineskip}

  Berlin, den 29. Mai 2018\\[1.5\baselineskip]

  Gutachterinnen\slash Gutachter:\\[0.5\baselineskip]
  1.~Prof.\ Dr.\ Anke Lüdeling (Humboldt-Universität zu Berlin)\\[0.25\baselineskip]
  2.~Prof.\ Dr.\ Stefan Müller (Humboldt-Universität zu Berlin)\\[0.25\baselineskip]
  3.~Prof.\ Dr.\ Matthias Hüning (Freie Universität Berlin)\\
}

\justifying
\rmfamily

\newpage
\thispagestyle{empty}

\section*{Erklärung über Zusammenarbeit mit anderen Wissenschaftlern\slash Wissenschaftlerinnen und den eigenen Anteil an der vorgelegten Leistung}

Von den vier Zeitschriftenartikeln, die in der hier vorgelegten kumulativen Habilitationsschrift zusammengefasst wurden, entstanden die folgenden beiden in Zusammenarbeit mit Frau Dr.\ Ulrike Sayatz (Deutsche und niederländische Philologie, Freie Universität Berlin):

\vspace{1\baselineskip}

\begin{enumerate}[label=\arabic*.]
  \item Schäfer, Roland \& Ulrike Sayatz.\ 2014.\ Die Kurzformen des Indefinit\-artikels im Deutschen.\ Zeitschrift für Sprachwissenschaft 33(2).\ 215--250.
  \item Schäfer, Roland \& Ulrike Sayatz.\ 2016.\ Punctuation and syntactic structure in ``obwohl'' and ``weil'' clauses in nonstandard written German.\ Written Language and Literacy 19(2).\ 212--245.
\end{enumerate}

\vspace{1\baselineskip}

\noindent Ich erkläre hiermit für die beiden oben genannten Artikel gleichermaßen:

\vspace{1\baselineskip}

\begin{enumerate}[label=\arabic*.]
  \item Die Aufarbeitung der bestehenden Literatur erfolgte zu gleichen Teilen durch Frau Dr.\ Sayatz und mich.
  \item Die theoretische Einordnung und die Hypothesenbildung erfolgte in Diskussionen mit Frau Dr.\ Sayatz.
    Das schriftliche Ergebnis dieser Diskussionen geht zu ungefähr einem von drei Teilen auf Frau Dr.\ Sayatz und zu zwei von drei Teilen auf mich zurück.
  \item Die empirische Forschung wurde vollumfänglich und eigenverantwortlich von mir durchgeführt.
  \item Die Niederschrift erfolgte vollumfänglich durch mich.
\end{enumerate}

\vspace{1\baselineskip}

\noindent Alle anderen Teile der vorgelegten Habilitationsschrift sind vollumfänglich durch meine eigene Leistung entstanden.

\vspace{1.5\baselineskip}

\noindent Berlin, den 29. Mai 2018\\[1.5\baselineskip]

\noindent \underline{\hspace{8cm}}\\[0.5\baselineskip]
\noindent Dr. Roland Schäfer

\newpage

\setcounter{page}{1}

\tableofcontents

\section*{Attachments (published papers)}
\begin{enumerate}
  \item \RODefArt
  \item \ROWeakN
  \item \ROMeasure
  \item \ROWeil 
\end{enumerate}

\newpage

\section*{Acknowledgements}

First and foremost, I want to thank Ulrike Sayatz, without whom the work presented here would not exist.
Among other things, Ulrike convinced me that graphemics is a highly relevant discipline and an integral part of linguistics and grammar.
Together, we developed the ideas of usage-based graphemics, which underlie some of the work presented here.
Ulrike also helped me to understand German morphosyntax much better through many discussions.

Furthermore, I am grateful to Felix Bildhauer for our joint work on the COW corpora (the data base underlying all of my research presented here) as well as for ongoing discussions about corpus construction, corpus analysis, and statistics (worth a substantial amount of Złoty).

I also thank Elizabeth Pankratz for being an inspiring collaborator, co-author, and moral supporter.
(Not to mention proofreader under difficult circumstances.)

Moreover, I want to express my highest gratitude to (in alphabetical order) Matthias Hüning, Anke Lüdeling, and Stefan Müller for agreeing to act as referees in the official process of my \textit{Habilitation} at \textit{Humboldt-Universität zu Berlin}.

For their support as teachers, colleagues, employers, enablers, thesis advisors, friends, moral supporters -- or any combination thereof -- throughout my career, I am indebted to (in alphabetical order) Dirk Buschbom, Susanne Flach, Thomas M.\ Groß, Iris Hasselberg, Götz Keydana, Michael Job, Stefan Müller, Bjarne Ørsnes, Erich Poppe, Manfred Sailer, Nicolai Sinn, and Gert Webelhuth.

Student assistants who contributed significantly to the success of my research by doing a lot of painstaking work on corpus data and supervising experiments are (in alphabetical order) Sarah Dietzfelbinger, Lea Helmers, Kim Maser, and Luise Rißmann.

Finally, I want to thank everyone who supported me on a personal level (and put up with my quirks and mannerisms) during my life in the academia so far, including but surely not restricted to (in alphabetical order) Matthias B.\ Döring, Michael Karg, Tanja Hagedorn, my parents, and Julia Schmidt.

\newpage

\section*{Preface}

\begin{quote}
  Thus, [what we do] may, because of the neglect of other important structural properties, be to classify natural language along an ultimately irrelevant dimension. \citep[436--437]{ParteeEa1990}
\end{quote}

I used the same quote from Partee, ter Meulen, and Wall's introduction to \textit{mathematical methods in linguistics} in the preface to my doctoral dissertation \citep{Schaefer2010}.
In their book, the sentence alerts readers that the Chomsky hierarchy and the theory of automata might not be an adequate framework for the description of human language.
In my dissertation, I used it to articulate my doubts that predicate logic (with event ontologies) and lambda calculus are adequate frameworks for the description of linguistic meaning, advocating a purely set theoretic description of sentence meaning.
In the present context, I reuse the quote to mark two aspects of linguistic theory which seem important to me at the present point in the field's development.

First, in recent decades, evidence has been collected which points to the fact that language is more of a probabilistic phenomenon (where rule application is a random process governed by chance and weighted lexical, grammatical, and contextual factors) than linguists thought before the 1990s (see, for example, the programmatic paper by \citealt{Bresnan2007}).
Prominently, among the influencing factors are even item-specific frequency-driven effects such as the co-occurrence affinities of words to each other (as examined in the much older tradition of collocation research; \citealt{Evert2008}) as well as co-occurrence affinities of words and constructions \citep{StefanowitschGries2003,GriesStefanowitsch2004,Gries2015b}.
This begs the questions of whether and how language users do not only learn rules (or \textit{generalisations}, to use a more neutral term) but also probability distributions over rules\slash generalisations and lexical items, \ie\ whether the probabilistic nature of language as used is part of the linguistic knowledge or can be traced to performance effects.
Since the work presented here consists of explorations in probabilistic German morphosyntax, I use the quote above to mark my belief that traditional grammatical modelling might be deficient in some relevant way.
Section~\ref{sec:probabilisticgrammar} discusses this in some detail, taking a careful stance and avoiding far-reaching claims about the architecture of the human language faculty.

Second, I find it particularly interesting that \citet{ParteeEa1990} was called an introduction to \textit{mathematical methods in linguistics}, and that many (but by no means all) present-day readers would expect something completely different under this label.
In the textbook, the relevant mathematical methods are considered to be theoretical algebra, set theory, systems of logic, the theory of automata, etc., while many of today's readers might expect statistics from such a volume.
At the time, quantitative statistical methods were not widely used in linguistics, except maybe in experimental psycholinguistics and some strains of sociolinguistics.%
\footnote{Also, in some functional\slash cognitive circles, small-scale corpus studies were analysed using simple statistics for counts since the 1980s; see \citet[8]{Gries2017a}.}
Grammar (comprising at least phonology, morphology, syntax, and referential semantics) was not seen as requiring a stochastic approach, and statistics was not part of most linguistic curricula.
Thus, by taking the quote out of its original context, I want to highlight the fact that statistical analysis and statistical modelling might now be on their way to becoming \textit{mathematical methods in linguistics} which are just as important as algebra, set theory, and the theory of automata.
Ideally, statistical modelling should eventually go far beyond the use of statistics in the analysis of results obtained from corpus studies and psycholinguistic experiments, leading to integrated stochastic models of language which require knowledge of all kinds of \textit{mathematical methods} (for example, \citealt{Bod2006}).
While linguistics as a discipline is clearly on its way to such an approach, a lot more theoretical and empirical work is still required.

The work presented here is theory-driven but mainly empirical.
In this work, I use the methods of probabilistic grammar, specifically the now-standard methods of alternation modelling.
While a lot of work exists on English alternations, German can be said to be under-researched in this kind of alternation modelling.
This is surprising considering the fact that German is famous for its numerous so-called grammatical \textit{Zweifelsfälle} `cases of doubt' \citep{Klein2009,Duden09}, which are nothing but alternations between equally acceptable forms and constructions.
These phenomena are ideal test cases for probabilistic approaches.

I also present some work in which I contribute to gauging the importance and the relation between corpus data (my main source of data) and psycholinguistic experiments.
Furthermore, my work makes methodological contributions by advancing relatively new sources of data (mainly web corpora), analysing non-standard language, and using and evaluating state-of-the-art statistical methods.
Section~\ref{sec:theoriesmethodsanddata} deals with the methodological issues in detail.
While no strict formal systems have been established for the modelling of the observed effects, my work contributes to defining and delimiting the requirements to be met by future integrated formal systems of language as represented in the minds of language users.
I consider it of great importance to gather data in a methodologically sound way -- as opposed to rushing linguistic theory towards another battle of frameworks (see Section~\ref{sec:probabilisticgrammar}).

\begin{figure}[htpb]
  \centering
  \includegraphics[width=\textwidth]{graphics/langsuse}
  \caption{Languages covered in the three major corpus linguistics journals; \textit{None} was assigned for papers which only address general or theoretical issues without reporting any original empirical work; (\textit{English}) was assigned to papers where English is used for comparison in papers predominantly about other languages; 35 languages which only occurred once are not shown}
  \label{fig:langsuse}
\end{figure}

There is one final point I would like to make right at the outset.\label{abs:survey}
My research on German was mostly published in international journals (such as Corpus Linguistics and Linguistic Theory [CLLT] and Cognitive Linguistics [COGL]).
The international corpus linguistics scene is very active, with at least three major journals (International Journal of Corpus Linguistics [IJCL], Corpora, CLLT) publishing large numbers of papers per year.
In 2016, in preparation for \citet{Schaefer2019}, an open-access introduction to statistical inference and statistical modelling for linguists, I performed a manual annotation of all 198 papers published in IJCL, CLLT, and Corpora between 2010 and 2015.%
\footnote{The raw data will be published with the book.}
The list of languages covered, the corpora used, and the statistical methods used in each paper were annotated.
Figure~\ref{fig:langsuse} shows the distribution of languages (as raw counts).%
\footnote{Since some papers deal with more than one language, 236 language codes were assigned in total.}

English featured prominently in 136 papers (146 if World Englishes and English sign languages are added), followed by Spanish with eleven mentions as a distant second.
German, on the other hand, was a major object of study in only seven papers (four in IJCL, two in Corpora, one in CLLT).
Of course, this does not mean that linguists working on German (or Spanish, Chinese, Dutch, French, etc.\ for that matter) do not use corpora or do not publish their research.
However, this result shows how corpus linguistics as a field is still very much identified with English corpus linguistics (or even BNC linguistics, see Section~\ref{sec:corporaincognitivelyorientedlinguistics}, esp.\ Figure~\ref{fig:corpususe} on p.~\pageref{fig:corpususe}).
While this state of affairs is not detrimental for corpus linguistics, I suggest that corpus linguists working on languages other than English could benefit from taking part in the active theoretical and methodological discussions taking place in international journals.
From my own point of view as a linguist working on German, it seems evident that the German language and German linguistics has a lot to contribute to current debates in corpus linguistics, especially given that German is famous for the probabilistic phenomena labelled \textit{cases of doubt}.
Thus, I hope my work encourages other linguists working on German (and other under-represented languages) to increase the visibility of their object of study in international corpus linguistics for mutual benefit.
While the case studies focus strongly on German grammar, this general introduction predominantly takes up the foundational theoretical and methodological issues.

\newpage

\chapter{Probabilistic grammar}
\label{sec:probabilisticgrammar}

All case studies presented here are empirical explorations of alternation phenomena in the broad sense.
While the term \textit{alternation} is sometimes reserved for syntactic (\ie\ constructional) alternations \citep{Gries2017a}, the four case studies deal with a morphographemic alternation in the context of the development of a new paradigm of the German indefinite article \RDefArt, a morphosyntactic alternation of so-called weak nouns which are gradually shifting towards another declension paradigm \RWeakN, a syntactic alternation between measure noun constructions \RMeasure, and a phenomenon at the syntax-graphemics interface where non-standard punctuation is an obvious indicator of different clausal connections \RWeil.

The nature of an alternation as understood here is that language users have different forms, constructions, or even paradigms at their disposal in a given situation of language production, and that they always make a choice (unless, of course, they decide not to make the utterance).%
\footnote{For the present purpose, I understand \textit{utterance} as comprising events of language production in both the spoken and the written mode.}
While in many cases, prosodic, syntactic, lexical, pragmatic, contextual, and other constraints can be found which account for why speakers tend to choose one form or the other, these constraints appear to be soft and to interact in a weighted fashion, and there often seems to be residual free variation in speakers' choices.
The observable phenomenon is thus clearly probabilistic or stochastic (as opposed to deterministic), and researchers have for a long time acknowledged this fact both for morphological phenomena (see the early overview in \citealt{HayBaayen2005}) and (morpho)syntactic phenomena (see early contributions such as \citealt{Gries2003,Wulff2003,Bresnan2007,Bresnan2007}).%
\footnote{While the number of studies which have produced empirical evidence for the probabilistic nature of morphology and (morpho)syntax is growing, it should be noted that graphemics is under-researched in this paradigm.
Writing is often not viewed as part of grammar or linguistics, and those who do advanced research on writing often view it under an acquisition perspective.
Since I cannot see why phonology and phonetics -- dealing with utterances realised in the spoken medium -- should be treated as part of grammar but graphemics -- dealing with utterances realised in the written medium -- should not (see also \citealt[495--500]{Schaefer2016e}), I extend the probabilistic view to graphemics.
The framework of \textit{usage-based graphemics} was therefore developed by Ulrike Sayatz and me in \ROWeil.
}

While my research clearly stands in the tradition of probabilistic grammar, I want to voice some concerns about the epistemological status of the evidence which we are gathering.
In usage-based and constructionist settings, the type of evidence as found in the papers presented here is often taken as supporting a model of grammar that does without the Chomskyan separation of competence and performance (\citealt{Chomsky1965}; an overview can be found in \citealt[507--518]{Mueller2018}) and\slash or does not embrace an algebraic Aristotelian theory of language in terms of discrete linguistic categories (\eg\ \citealt{Manning2002,Bod2006}; see also \citealt{Kapatsinski2014} for a recent and subjective overview in the same vein).
First of all, doing away with performance altogether is clearly not a reasonable approach considering the body of psycholinguistic research showing how processing constraints affect speakers' and hearers' language use depending on factors clearly not related to learned generalisations (be they stochastic or not).
As \citet[532]{Pullum2013a} puts it,

\begin{quote}
  no sensible grammarian wants or expects grammars to yield direct representations of the raw reality of human linguistic behaviour with all its flubs, false starts and lost trains of thought.
\end{quote}

The core question is rather whether linguistic competence itself is a probabilistic (non-Aristotelian) system (as claimed programmatically by \citealt{Bresnan2007}) or whether probabilistic effects arise from performance alone.%
Another important and only partially related question is whether linguistic knowledge is highly specified, isolated from other forms of knowledge (including linguistic semantic knowledge as distinct from encyclopedic knowledge), and possibly also modular-serial \citep{Fodor1995}, or linguistic knowledge (including semantics) is connected to and (mostly) an indistinguishable part of non-linguistic knowledge (see \citealt{Elman2009}).
While I personally favour a non-Aristotelian view which does not arbitrarily ascribe phenomena to performance, nothing in the data presented in the probabilistic tradition (which includes alternation research) conclusively forces us to assume any specific architecture as is sometimes assumed in usage-based circles \cite{BybeeBeckner2009}.
The traditional division of labour between competence and performance, together with a model-theoretic algebraic theory of syntax including an appropriate probabilistic (constraint-weighting) component, could in principle model all observed effects, including graded acceptability, stochastic alternations, and context-driven effects (\eg\ \citealt[504--507]{Pullum2013}, \citealt{Pullum2013a}, and \citealt[499--500,507--518]{Mueller2018}).%
\footnote{It must be noted, however, that while proponents of model-theoretic syntax often suggest that a probabilistic version is possible \citep[500]{Mueller2018} or even trivial, only few (and often sketchy and inconsequential) attempts have been made to deliver actual implementations (\eg\ \citealt{ArnoldLindarki2007}).}
While some frameworks might make it (apparently) easier to model stochastic effects (sometimes at the cost of coverage or rigidity of formalisation), evidence which decides between theory A -- which assumes highly specific deterministic linguistic knowledge in combination with performance effects and separate from contextual encyclopedic world knowledge -- and theory B -- which favours a stochastic version of linguistic knowledge, not as cleanly separated from encyclopedic knowledge -- is hard to come by.

Interestingly, \citet{Elman2009} (who is on the very far non-Aristotelian end of the Aristotelian vs.\ non-Aristotelian continuum) proposes a radical connectionist model (based on ample experimental evidence and computational models) which does away with the mental lexicon, a component which is in some way, shape, or form part of virtually any linguistic theory in the narrow sense (including stochastic, usage-based, constructionist theories).
In simple terms, Elman proposes a model where words do not have semantic content but merely serve as cues to conceptual and world knowledge.
Despite his comprehensive and erudite argumentation, he admits that the evidence and the computational models are in no way conclusive evidence for his approach \citep[573--574]{Elman2009}.
\citet[573]{Elman2009} states:

\begin{quote}
  However, theories can also be evaluated for their ability to offer new ways of thinking about old problems, or to provoke new questions that would not be otherwise asked.
  A theory might be preferred over another because it leads to a research program that is more productive than the alternative.
\end{quote}

This statement reminds one of the Kuhnian view of \textit{normal science} as a state where a research programme generates enough new and exciting \textit{puzzles} for researchers to solve in order to keep the field alive \citep{Kuhn1970}.
As a matter of fact, the research on alternations and other stochastic phenomena has thrived in frameworks (such as cognitive, usage-based, constructionist linguistics) which try do away (as much as possible) with the competence-performance dichotomy and with a highly modular and specific model of linguistic competence and \textit{not} in generative frameworks (such as Chomskyan minimalism) or model-theoretic frameworks (such as Head-Driven Phrase Structure Grammar), which is why the term \textit{probabilistic linguistics} has become associated with the former type of framework.
My research is therefore presented with reference to usage-based approaches, prominently addressing the prototype vs.\ exemplar debate.%
\footnote{However, it does without a specific commitment to constructionist approaches or cognitive grammar in the narrow sense of \citet{Langacker1987}.}
In a spirit similar to the Jeffrey Elman quote above, I consider the usage-based framework and the associated community the one which currently \textit{offers new ways of thinking about old problems, and which provokes new questions that would not be otherwise asked}, or, in Kuhnian terms, which \textit{offers enough new puzzles to solve} when it comes to stochastic surface effects in language use.
In no way does this mean that I consider the empirical findings intrinsically incompatible with other frameworks.
As soon as people from such frameworks develop a significant interest in modelling my data, they may do so.%
\footnote{Such attempts would be facilitated by the fact that I publish all data related to my published research freely (see \url{https://github.com/rsling}).}

All that said, I see my work less as speaking in favour of any specific linguistic framework and more as making theoretic and above all methodological contributions to some very specific questions, such as the prototype vs.\ exemplar debate (in \RAWeakN, \RAWeil, and \RAMeasure), paradigmatic morphology (in \RADefArt\ and \RAWeakN), graphemics under a usage-based perspective (in \RAWeil), the experimental cross-validation of corpus-derived models (in \RAMeasure).%
\footnote{With respect to the prototype vs.\ exemplar debate, however, a similar situation is described for corpus linguistics in Section~\ref{sec:prototypesandexemplars}.
The data only provide limited cues as to which theory is more appropriate.}
All case studies promote the use of web corpora as ideal sources of data, which is argued for in Section~\ref{sec:corporaincognitivelyorientedlinguistics}.
Finally, the statistical methods used in the analysis of corpus and experimental data are one of my primary foci, which is why Section~\ref{sec:statistics} provides a short overview of statistics and scientific inference.


\newpage

\input{sections/2theoriesmethodsanddata}

\newpage

\input{sections/3casestudies}

\newpage

\section{Future directions}
\label{sec:futuredirections}

My and my co-author's research collected here shows that German has a wide range of phenomena to offer for examination under a usage-based probabilistic perspective.
Furthermore, in the form of the DECOW corpus, a now-proven source of data exists which allows researchers to work on these phenomena.
Based on the argumentation in Sections~\ref{sec:probabilisticgrammar}--\ref{sec:casestudies} and the case studies, a number of open research questions come to mind.
I see at least the following ones.

\vspace{\baselineskip}

\begin{itemize}
  \item The literature on \textit{cases of doubt} in German is famously rich.
    It would be beneficial for linguists working on German, corpus linguists, and linguists working in the cognitively oriented\slash usage-based tradition to examine them using the framework established here.
  \item The effects of corpus composition and the availability of metadata on corpus samples and sampling procedures should be examined further.
    The BNC is rich in metadata and has a well-planned composition, but for many other languages (like German), similar corpora do not exist.
  \item Related to the last point, corpora containing non-standard writing should be honoured more as a unique source of data.
    While there is a community working on such corpora and specific analyses of their content, many more (corpus) linguists could benefit from using them in the same way I did.
  \item Also related to this point, the usage-based perspective on graphemics as developed by Ulrike Sayatz and me should be developed and expanded further.
    Speakers' writing behaviour provides important clues to how they cognitively represent morphological and syntactic categories.
  \item Individual grammatical differences urgently require more attention.
    While it will probably be impossible to build large enough general-purpose corpora with speaker metadata which would allow research on individual grammatical differences, corpus data should be correlated with the reactions of individual speakers in controlled experiments.
  \item The prototype vs.\ exemplar debate would benefit from more large-scale corpus studies which must then be cross-validated in controlled experiments.
    Corpus data alone cannot provide evidence for or against one theory or the other.
  \item Statistical methods need to be scrutinised.
    While mindless applications of NHST are detrimental for valid scientific inferences, some critiques of frequentist statistics (language is never random; model everything) have gone too far or are understood in a much too unrestricted manner.
    Also, some currently-hyped alternative methods do not lead to substantially different results, are understood even less than traditional methods, and distract from the real problems with statistical inference.
\end{itemize}

\vspace{\baselineskip}

Clearly, my work has contributed to all of these points, but the overall situation in probabilistic, usage-based, cognitively oriented corpus linguistics is one where methods and theories are still in a very early stage of development.


\newpage

\printbibliography

\end{document}
