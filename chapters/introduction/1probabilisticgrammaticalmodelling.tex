\section{Probabilistic grammar}
\label{sec:probabilisticgrammar}

All case studies presented here are empirical explorations of alternation phenomena in the broad sense.
While the term \textit{alternation} is sometimes reserved for syntactic (\ie\ constructional) alternations \citep{Gries2017a}, the four case studies deal with a morphographemic alternation in the context of the development of a new paradigm of the German indefinite article \RDefArt, a morphosyntactic alternation of so-called weak nouns which are gradually shifting towards another declension paradigm \RWeakN, a syntactic alternation between measure noun constructions \RMeasure, and a phenomenon at the syntax-graphemics interface where non-standard punctuation is an obvious indicator of different clausal connections \RWeil.

The nature of an alternation as understood here is that language users have different forms, constructions, or even paradigms at their disposal in a given situation of language production, and that they always make a choice (unless, of course, they decide not to make the utterance).%
\footnote{For the present purpose, I understand \textit{utterance} as comprising events of language production in both the spoken and the written mode.}
While in many cases, prosodic, syntactic, lexical, pragmatic, contextual, and other constraints can be found which account for why speakers tend to choose one form or the other, these constraints appear to be soft and to interact in a weighted fashion, and there often seems to be residual free variation in speakers' choices.
The observable phenomenon is thus clearly probabilistic or stochastic (as opposed to deterministic), and researchers have for a long time acknowledged this fact both for morphological phenomena (see the early overview in \citealt{HayBaayen2005}) and (morpho)syntactic phenomena (see early contributions such as \citealt{Gries2003,Wulff2003,Bresnan2007,Bresnan2007}).%
\footnote{While the number of studies which have produced empirical evidence for the probabilistic nature of morphology and (morpho)syntax is growing, it should be noted that graphemics is under-researched in this paradigm.
Writing is often not viewed as part of grammar or linguistics, and those who do advanced research on writing often view it under an acquisition perspective.
Since I cannot see why phonology and phonetics -- dealing with utterances realised in the spoken medium -- should be treated as part of grammar but graphemics -- dealing with utterances realised in the written medium -- should not (see also \citealt[495--500]{Schaefer2016e}), I extend the probabilistic view to graphemics.
The framework of \textit{usage-based graphemics} was therefore developed by Ulrike Sayatz and me in \ROWeil.
}

While my research clearly stands in the tradition of probabilistic grammar, I want to voice some concerns about the epistemological status of the evidence which we are gathering.
In usage-based and constructionist settings, the type of evidence as found in the papers presented here is often taken as supporting a model of grammar that does without the Chomskyan separation of competence and performance (\citealt{Chomsky1965}; an overview can be found in \citealt[507--518]{Mueller2018}) and\slash or does not embrace an algebraic Aristotelian theory of language in terms of discrete linguistic categories (\eeg\ \citealt{Manning2002,Bod2006}; see also \citealt{Kapatsinski2014} for a recent and subjective overview in the same vein).
First of all, doing away with performance altogether is clearly not a reasonable approach considering the body of psycholinguistic research showing how processing constraints affect speakers' and hearers' language use depending on factors clearly not related to learned generalisations (be they stochastic or not).
As \citet[532]{Pullum2013a} puts it,

\begin{quote}
  no sensible grammarian wants or expects grammars to yield direct representations of the raw reality of human linguistic behaviour with all its flubs, false starts and lost trains of thought.
\end{quote}

The core question is rather whether linguistic competence itself is a probabilistic (non-Aristotelian) system (as claimed programmatically by \citealt{Bresnan2007}) or whether probabilistic effects arise from performance alone.%
Another important and only partially related question is whether linguistic knowledge is highly specified, isolated from other forms of knowledge (including linguistic semantic knowledge as distinct from encyclopedic knowledge), and possibly also modular-serial \citep{Fodor1995}, or linguistic knowledge (including semantics) is connected to and (mostly) an indistinguishable part of non-linguistic knowledge (see \citealt{Elman2009}).
While I personally favour a non-Aristotelian view which does not arbitrarily ascribe phenomena to performance, nothing in the data presented in the probabilistic tradition (which includes alternation research) conclusively forces us to assume any specific architecture as is sometimes assumed in usage-based circles \cite{BybeeBeckner2009}.
The traditional division of labour between competence and performance, together with a model-theoretic algebraic theory of syntax including an appropriate probabilistic (constraint-weighting) component, could in principle model all observed effects, including graded acceptability, stochastic alternations, and context-driven effects (\eeg\ \citealt[504--507]{Pullum2013}, \citealt{Pullum2013a}, and \citealt[499--500,507--518]{Mueller2018}).%
\footnote{It must be noted, however, that while proponents of model-theoretic syntax often suggest that a probabilistic version is possible \citep[500]{Mueller2018} or even trivial, only few (and often sketchy and inconsequential) attempts have been made to deliver actual implementations (\eeg\ \citealt{ArnoldLindarki2007}).}
While some frameworks might make it (apparently) easier to model stochastic effects (sometimes at the cost of coverage or rigidity of formalisation), evidence which decides between theory A -- which assumes highly specific deterministic linguistic knowledge in combination with performance effects and separate from contextual encyclopedic world knowledge -- and theory B -- which favours a stochastic version of linguistic knowledge, not as cleanly separated from encyclopedic knowledge -- is hard to come by.

Interestingly, \citet{Elman2009} (who is on the very far non-Aristotelian end of the Aristotelian vs.\ non-Aristotelian continuum) proposes a radical connectionist model (based on ample experimental evidence and computational models) which does away with the mental lexicon, a component which is in some way, shape, or form part of virtually any linguistic theory in the narrow sense (including stochastic, usage-based, constructionist theories).
In simple terms, Elman proposes a model where words do not have semantic content but merely serve as cues to conceptual and world knowledge.
Despite his comprehensive and erudite argumentation, he admits that the evidence and the computational models are in no way conclusive evidence for his approach \citep[573--574]{Elman2009}.
\citet[573]{Elman2009} states:

\begin{quote}
  However, theories can also be evaluated for their ability to offer new ways of thinking about old problems, or to provoke new questions that would not be otherwise asked.
  A theory might be preferred over another because it leads to a research program that is more productive than the alternative.
\end{quote}

This statement reminds one of the Kuhnian view of \textit{normal science} as a state where a research programme generates enough new and exciting \textit{puzzles} for researchers to solve in order to keep the field alive \citep{Kuhn1970}.
As a matter of fact, the research on alternations and other stochastic phenomena has thrived in frameworks (such as cognitive, usage-based, constructionist linguistics) which try do away (as much as possible) with the competence-performance dichotomy and with a highly modular and specific model of linguistic competence and \textit{not} in generative frameworks (such as Chomskyan minimalism) or model-theoretic frameworks (such as Head-Driven Phrase Structure Grammar), which is why the term \textit{probabilistic linguistics} has become associated with the former type of framework.
My research is therefore presented with reference to usage-based approaches, prominently addressing the prototype vs.\ exemplar debate.%
\footnote{However, it does without a specific commitment to constructionist approaches or cognitive grammar in the narrow sense of \citet{Langacker1987}.}
In a spirit similar to the Jeffrey Elman quote above, I consider the usage-based framework and the associated community the one which currently \textit{offers new ways of thinking about old problems, and which provokes new questions that would not be otherwise asked}, or, in Kuhnian terms, which \textit{offers enough new puzzles to solve} when it comes to stochastic surface effects in language use.
In no way does this mean that I consider the empirical findings intrinsically incompatible with other frameworks.
As soon as people from such frameworks develop a significant interest in modelling my data, they may do so.%
\footnote{Such attempts would be facilitated by the fact that I publish all data related to my published research freely (see \url{https://github.com/rsling}).}

All that said, I see my work less as speaking in favour of any specific linguistic framework and more as making theoretic and above all methodological contributions to some very specific questions, such as the prototype vs.\ exemplar debate (in \RAWeakN, \RAWeil, and \RAMeasure), paradigmatic morphology (in \RADefArt\ and \RAWeakN), graphemics under a usage-based perspective (in \RAWeil), the experimental cross-validation of corpus-derived models (in \RAMeasure).%
\footnote{With respect to the prototype vs.\ exemplar debate, however, a similar situation is described for corpus linguistics in Section~\ref{sec:prototypesandexemplars}.
The data only provide limited cues as to which theory is more appropriate.}
All case studies promote the use of web corpora as ideal sources of data, which is argued for in Section~\ref{sec:corporaincognitivelyorientedlinguistics}.
Finally, the statistical methods used in the analysis of corpus and experimental data are one of my primary foci, which is why Section~\ref{sec:statistics} provides a short overview of statistics and scientific inference.
