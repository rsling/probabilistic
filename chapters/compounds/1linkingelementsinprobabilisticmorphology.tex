\section{Linking elements in probabilistic morphology}
\label{sec:linkingelementsinprobabilisticmorphology}

In this paper, we examine so-called \textit{linking elements} in German nominal compounds.
Linking elements are optional segmental material inserted between the first and second nominal constituents of a compound, such as \textit{-er} in \textit{Liedertext} `lyrics', literally `song text'.
Linking elements are not obligatory, and many compounds are acceptable with or without them.
For instance, the variant \textit{Liedtext} with zero linking element also exists, though speakers may have individual preferences for compounds with or without linking elements (as we will discuss further below).
The distribution of the relatively large number of linking elements has previously been described in terms of constraints (some of them soft, others firmly categorical) from different angles (\egg \citealt{Fuhrhop1996,Wegener2003,Schluecker2012,NueblingSzczepaniak2013,FuhrhopKuerschner2015}).
Conceptually complete models of linking element selection have been proposed in rule-based systems (possibly with a similarity-based component; see \citealt{DresslerEa2001}) and entirely exemplar-based models \parencite{KrottEa2007}.

It is often assumed that linking elements have neither a grammatical function nor a semantic interpretation.
In this paper, we question this perspective and look more closely at the possibility that certain linking elements could have a plural interpretation.
Interestingly, except for several very rare linking elements, a default zero linking element, and the linking element \textit{-(e)s}, all linking elements are formally identical to the first constituent's plural marker.
Additionally, many first constituents alternate between such a pluralic linking element and another non-pluralic linking element (typically zero or \textit{-(e)s}).
This raises the question of whether speaker-hearers or writer-readers (or both) may associate a plural meaning with such pluralic linking elements.
So far, researchers have been skeptical toward or even dismissive of this possibility, and in a perception experiment, \textcite{KoesterEa2004} could not find evidence that a plural interpretation of linking elements is triggered in spoken German.
However, for (written) Dutch linking elements, such effects have been demonstrated repeatedly \parencite{SchreuderEa1998,BangaEa2012,BangaEa2013a,BangaEa2013b}.
In this paper, we present systematic research in the form of corpus data and the results of an experiment using the split-100 paradigm which shows that German linking elements are indeed used as cues for the plural meaning of the first constituent in written German compounds.

Findings like this have, in our view, an impact on morphological theory in general for two closely related reasons.
First, any morphological framework or theory must be flexible enough to be able to account for phenomena which have been demonstrated to exist -- in our case, the systematic occurrence of inflected forms within products of word-formation such as compounds.
Classical generative frameworks (\egg \citealt{Siegel1979,Mohanan1986,Anderson1992,Pinker1999}) tended to implement strong universal tendencies as hard constraints into the architecture of the framework, which, in our view, does not allow the required flexibility.
One example is Siegel's strict layering hypothesis, which states that inflection applies derivationally after word formation and that consequently, markers of inflectional categories can only be positioned at the edges of words (as we will discuss further in Section~\ref{sec:consequencesofplurallinkagesformorphologicaltheories}).
Such approaches have been criticised for many decades, since, as \textcite[391]{Haspelmath2010} puts it, ``most empirical universals are tendencies''.
\textcite{Pollard1996} astutely criticises the absurdity of needlessly restrictive generative frameworks in syntax, and \textcite{Haspelmath2010} goes even further by suggesting that frameworks should be abandoned altogether, demonstrating with many examples how frameworks sacrifice explanatory adequacy for hard-wired allegedly universal restrictions.
With respect to inflection inside German compounds, we provide a framework-free but differentiated image to be summarised in Section~\ref{sec:conclusion}.

Second, findings such as ours lend support to a probabilistic view of morphology and grammar in general.
The amount of evidence for the inherent gradedness of grammar has been growing for decades.
\textcite{HayBaayen2005} summarised an impressive number of studies about this topic as it concerns morphology, and \textcite{Bresnan2007} and subsequent work have radically changed the way empirical research is conducted in syntax.
Even the relevance for linking elements has not gone uncommented; \textcite[105]{ArndtlappeEa2016} include the selection of linking elements in their list of ``semi-systematic and gradient properties'' of compounds.
These properties have been actively researched in recent years, and \textcite{ArndtlappeEa2016} stress the important role played by empirical work in this area of investigation.
The present study contributes to this body of work by showing that there is a -- by no means categorical -- tendency for linking elements which take the form of a plural to be interpreted as plurals.
\textcite[107]{ArndtlappeEa2016} address precisely this issue when they state that ``although there is not and probably never has been a one-to-one correspondence between the form and meaning of compounds, the form does provide a wide variety of information to which humans have access in reaching an interpretation''.
As we will argue in Section~\ref{sec:conclusion}, it would even be surprising if writer-readers did not pick up on the possibility of using plural forms to denote plural where it makes perfect sense, at least in the absence of strong inhibitory factors.

The paper is structured as follows.
In Section~\ref{sec:linkingelementsingerman}, we review the form and distribution of linking elements in German, and we discuss positions about their potential plural interpretation from the literature.
This includes a discussion of a long tradition of research on Dutch linking elements and their plural interpretability.
The section closes with an outline of our hypotheses concerning when there might be a preference for pluralic linking elements in compounds, for instance when the internal conceptual structure of the compound enforces a plural interpretation of the first constituent.
Section~\ref{sec:data} describes a database we created based on the DECOW16A web corpus to investigate these hypotheses.
The database shows the frequencies with which a large number of nouns occurring as first constituents in compounds take pluralic and non-pluralic linking elements.
It also quantifies the productivity of each of these nouns with these linking elements.
In Section~\ref{sec:corpusstudy}, we use the database to select a large number of first constituents which exhibit linking element alternation (in other words, first constituents which occur reliably with both pluralic and non-pluralic linking elements) and show, using a manually annotated corpus sample of approximately $10,000$ compounds containing these first constituents, that pluralic linking elements are indeed cues for semantic plurality.
In Section~\ref{sec:split100experiment}, we report an experiment in the split-100 paradigm which corroborates our findings from the corpus study.
Finally, in Section~\ref{sec:conclusion}, we discuss the findings in the larger context of probabilistic morphology.
