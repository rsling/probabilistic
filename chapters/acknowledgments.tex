\addchap{Acknowledgments}

First and foremost, I want to thank Ulrike Sayatz.
Ulrike convinced me that graphemics is a highly relevant discipline and an integral and indispensible part of linguistics and grammar.
Together, we developed the ideas of \textit{usage-based graphemics}, which underlie some of the work presented here.
Ulrike also helped me to understand German morphosyntax much better through many discussions.
Most importantly, she has been a productive and inspiring co-author for many years now.

I thank Elizabeth Pankratz for being an inspiring collaborator, co-author and strong moral supporter.
Among other things, Elizabeth changed the way I look at compounds, leading to significant improvements in the DECOW corpus and fascinating joint research on compounds.

Furthermore, I am grateful to Felix Bildhauer for ongoing discussions about corpus construction, corpus analysis, and statistics (worth at least a gazillion Złoty).
Without our collaboration, the COW corpora would not exist, and I could not have conducted the research presented in this book.

For their substantial support throughout my career as teachers, colleagues, employers, thesis advisors, and moral supporters (or any combination thereof), I am indebted to (in alphabetical order) Götz Keydana, Michael Job, Stefan Müller, Manfred Sailer, Gert Webelhuth.
Without any single one of them, I would not have made it this far.

Student assistants who contributed significantly to the success of my research by doing a lot of painstaking work on corpus data and supervising experiments are (in alphabetical order) Sarah Dietzfelbinger, Lea Helmers, Kim Maser, and Luise Rißmann.
I am truly grateful to them.

Finally and most importantly, I want to thank everyone who supported me on a personal level over the years, including but surely not restricted to (in alphabetical order) Matthias B.\ Döring, Tanja Hagedorn, Julia Schmidt, and my parents.

My work on this project was funded in part by the \textit{Deutsche Forschungsgemeinschaft} (DFG, personal grant SCHA1916/1-1) through the project \textit{Linguistic Web Characterisation}.
